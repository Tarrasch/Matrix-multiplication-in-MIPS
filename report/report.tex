\documentclass[a4paper,11pt]{article}
\usepackage[T1]{fontenc}
\usepackage[utf8]{inputenc}
\usepackage{lmodern}
\usepackage{hyperref}

\title{Monetarily efficient Guass Elminiation given crappy hardware}
\author{Arash Rouhani and Jonas Ängeslevä}

\begin{document}

\maketitle

\begin{abstract}

\end{abstract}

\section{Cmoputer System}
We have written a gauss elimination program for MIPS assembler, the program is optimized for the hardware. Details below.
\subsection{hardware}
The MIPS processor have 5 pipe lines stages, seperate I and D cache, a write buffer and two coprocessors.
The caches can be in sizes of 128, 256 and 512 bytes, and are using LRU replacement policy and Write Back replacement policy.
The Least Recently Used is a replacement policy, where the element in the set that was least recently used is replaced, if the cache is full. Note that this only maters in N-way caches (where N>1).
The Write Back writing policy means that when writing to the memory, we don't actually write to the memory if the memory is cached. 
\subsection{software}
The algorithm used for the elimination is the one suggested.
Only some minor optimizations have been at an algorithmical level. We do not unncessisarily loop at the last row (it's always 0 0 ... 0 0 1).
This assumption makes our code to actually only work on invertable matrices.

For optimization, the code does use delayed branching slots, does not do indexing or call any subroutines. 
Furthermore it does not violate the MIPS conventions for subroutines, it however mutates a0, so the test program must be corrected accordingly.


\end{document}
